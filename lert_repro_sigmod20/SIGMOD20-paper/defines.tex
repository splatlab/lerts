%!TEX root =  main.tex
\makeatletter
\DeclareRobustCommand*\cal{\@fontswitch\relax\mathcal}
\makeatother

%\newcommand\ourparagraph[1]{\paragraph*{\bf{#1}}}
\newcommand{\ourparagraph}[1]{\vspace{0.07in}\noindent{\bf \boldmath #1.}} 
%\newcommand{\myparagraph}[1]{\paragraph{#1}\mbox{}\\}


\newcommand{\mymarginpar}[1]{\marginpar{\sf \scriptsize #1 }}

\newcommand{\marginparforfigure}[1]{ \hspace*{\fill}\mbox{\sf \scriptsize \hspace*{\fill} #1 }}


\newcommand{\vect}[1]{\left[#1\right]}
\newcommand{\set}[1]{\left\{#1\right\}}
\newcommand{\seq}[1]{\left(#1\right)}
\renewcommand{\epsilon}{\varepsilon}
\newcommand{\ceil}[1]{\left\lceil #1 \right\rceil}


%\setlength{\parindent}{2em}

%cqf related defines
\newcommand{\qf}{quotient filter\xspace}
\newcommand{\qfrs}{rank-and-select quotient filter\xspace}
\newcommand{\cqf}{counting quotient filter\xspace}
\newcommand{\cqfs}{counting quotient filters\xspace}
\newcommand{\cf}{cuckoo filter\xspace}
\newcommand{\bloomf}{Bloom filter\xspace}
\newcommand{\cbf}{counting Bloom filter\xspace}
\newcommand{\sbf}{spectral Bloom filter\xspace}
\newcommand{\rank}{\mbox{\sc rank}}
\newcommand{\select}{\mbox{\sc select}}

%popcorn related defines
\newcommand{\punt}[1]{}
\newcommand{\calU}{{\cal U}}
\newcommand{\calC}{{\cal C}}
\newcommand{\calB}{{\cal B}}
\newcommand{\calS}{{\cal S}}
\newcommand{\calF}{{\cal F}}
\newcommand{\calD}{{\cal D}}



\newcommand{\FP}{\textsc{False Positive}\xspace}
\newcommand{\FN}{\textsc{False Negative}\xspace}
\newcommand{\FPs}{\textsc{False Positives}\xspace}
\newcommand{\FNs}{\textsc{False Negatives}\xspace}
\newcommand{\OL}{\textsc{Online}\xspace}
\newcommand{\Scalable}{\textsc{Scalable}\xspace}
%\newcommand{\iq}{\textsc{iq}\xspace}
\newcommand{\oedp}{\ted problem\xspace}
\newcommand{\ted}{\textsc{TED}\xspace}

% \newcommand{\tedp}{\textsc{ted}\xspace}
% \newcommand{\oedpfp}{\textsc{oedpfp}\xspace}

% \newcommand{\aoedp}{\tedp}
% \newcommand{\oaedp}{\textsc{oaedp}\xspace}

% \newcommand{\taedp}{\textsc{taedp}\xspace}

\newcommand{\LERT}{LERT\xspace} % just use capital letters  
\newcommand{\ts}{time-stretch LERT\xspace}
\newcommand{\cs}{count-stretch LERT\xspace}
\newcommand{\mgl}{Misra-Gries LERT\xspace}
% JWB: when used as a compound adjective, the hyphen is necessary 
%      ("immediate-report LERT")
%      ("power-law distribution")
% JWB: when used as a noun, there's no hyphen 
%      ("This LERT does immediate reports")
%      ("The exponent of the power law.")
\newcommand{\ps}{immediate-report LERT\xspace}
\newcommand{\Ts}{Time-stretch LERT\xspace}
\newcommand{\Cs}{Count-stretch LERT\xspace}
\newcommand{\Ps}{Immediate-report LERT\xspace}
\newcommand{\lert}{leveled external-memory reporting table\xspace}
\newcommand{\lerts}{leveled external-memory reporting tables\xspace}
\newcommand{\Lert}{Leveled external-memory reporting table\xspace}
\newcommand{\Lerts}{Leveled external-memory reporting tables\xspace}

\newcommand{\gma}{2.5 ({N}/{M})^{\frac{1}{\theta-1}}}

%These are old ones and shouldn't be used. Keeping them for now so that the
%paper compiled without errors.
%\newcommand{\timestretch}{time-stretch filter\xspace}
%\newcommand{\popcorn}{popcorn filter\xspace}
%\newcommand{\pf}{popcorn filter\xspace}
%\usepackage{ntheorem}

\makeatletter
%\renewtheoremstyle{plain}% Adds automatic line break, if heading is too long
%  {\item{\theorem@headerfont ##1\ ##2\theorem@separator}{\bf.}~}
%  {\item{\theorem@headerfont ##1\ ##2\ (##3)\theorem@separator}{\bf.}~}
%\makeatother
%
%{\theoremheaderfont{\upshape\bfseries}
% \theorembodyfont{\normalfont\em}
%\newtheorem{definition}{Definition}}


\makeatletter
\def\@copyrightspace{\relax}
\makeatother


\newcommand{\defn}[1]       {{\textit{\textbf{\boldmath #1}}}}
\newcommand{\pparagraph}[1]{\vspace{0.07in}\noindent{\bf \boldmath #1.}} 
\newcommand{\poly}{\mbox{poly}}
\newcommand{\polylog}{\mbox{polylog}}
%%%% VARIABLE NAMES %%%%

\newcommand{\Ns}{N}
\newcommand{\Nk}{U}
%% MAB: do we want stream to be $S$,  $\calS$, or something else? 
\newcommand{\stream}{S}


\newif\iffull
\fullfalse
%\fulltrue


\date{}

% Uncomment to enable comments
\newcommand{\namedcomment}[3]{{\sf \scriptsize \color{#2} #1: #3}}
\renewcommand{\namedcomment}[3]{}
% Comment next line to remove comments
%\renewcommand{\namedcomment}[3]{{\sf \scriptsize \color{#2} #1: #3}}



\newcommand{\revtwo}[1]{{\color{blue}{#1}}}
\newcommand{\revfour}[1]{{\color{blue}{#1}}}
\newcommand{\revfive}[1]{{\color{blue}{#1}}}
\newcommand{\new}[1]{{\color{green!60!black}{#1}}}

\newcommand{\mab}[1]{\namedcomment{mab}{red}{#1}}
\newcommand{\mfc}[1]{\namedcomment{mfc}{purple}{#1}}
\newcommand{\jon}[1]{\namedcomment{jon}{red}{#1}}
\newcommand{\cindy}[1]{\namedcomment{cindy}{green!60!black}{#1}}
\newcommand{\prashant}[1]{\namedcomment{prashant}{magenta}{#1}}
\newcommand{\todo}[1]{\namedcomment{todo}{blue}{#1}}
\newcommand{\tom}[1]{\namedcomment{tom}{purple}{#1}}
\newcommand{\shikha}[1]{\namedcomment{shikha}{blue}{#1}}
\newcommand{\rob}[1]{\namedcomment{rob}{orange}{#1}}
\newcommand{\robj}[1]{\namedcomment{rob}{orange}{#1}}
\newcommand{\task}[1]{\namedcomment{Task assigned}{red}{#1}}
% \begin{animateinline}[autoplay,loop]{2}%
%%   \randomcolor{randomcolora}
%%   \randomcolor{randomcolorb}
%%   \randomcolor{randomcolorc}

%%   \noindent\fadingtext{left color=randomcolora,right color=randomcolorb,middle color=randomcolorc!80!black}
%%              {\sf \scriptsize \sloppy \parbox{6.5in}{Rob:  #1}}
%%              %\newframe
%%              %\end{animateinline}
%% }
\newcommand{\varK}{24\xspace}
\renewcommand{\epsilon}{\varepsilon}
\newcommand{\bet}{B$^{\varepsilon}$-tree\xspace}
\newcommand{\bets}{B$^{\varepsilon}$-trees\xspace}

%% References
\newcommand{\appref}[1]         {Appendix~\ref{app:#1}}
\newcommand{\applabel}[1]    {\label{app:#1}}

\newcommand{\chapref}[1]        {Chapter~\ref{chap:#1}}
\newcommand{\secref}[1]         {Section~\ref{sec:#1}}
\newcommand{\seclabel}[1]    {\label{sec:#1}}
\newcommand{\subsecref}[1]      {Subsection~\ref{subsec:#1}}
\newcommand{\subseclabel}[1]    {\label{subsec:#1}}
\newcommand{\secreftwo}[2]      {Sections \ref{sec:#1} and~\ref{sec:#2}}
\newcommand{\secrefthree}[3]    {Sections \ref{sec:#1}, \ref{sec:#2}, and \ref{sec:#3}}
\newcommand{\secreffour}[4]     {Sections \ref{sec:#1}, \ref{sec:#2}, \ref{sec:#3}, and~\ref{sec:#4}}
\newcommand{\quanlabel}[1] {\label{quan:#1}}
\newcommand{\quanref}[1]  {Quantity~\ref{quan:#1}}
\newcommand{\figlabel}[1]   {\label{fig:#1}}
\newcommand{\figref}[1]         {Figure~\ref{fig:#1}}
\newcommand{\figreftwo}[2]      {Figures \ref{fig:#1} and~\ref{fig:#2}}
\newcommand{\tabref}[1]         {Table~\ref{tab:#1}}
\newcommand{\tablabel}[1]   {\label{tab:#1}}
\newcommand{\stref}[1]          {Step~\ref{st:#1}}
\newcommand{\thmlabel}[1]   {\label{thm:#1}}
\newcommand{\thmref}[1]         {Theorem~\ref{thm:#1}}
\newcommand{\thmabbrevref}[1]         {Thm.~\ref{thm:#1}}
\newcommand{\claimlabel}[1]         {\label{claim:#1}}
\newcommand{\claimref}[1]         {Claim~\ref{claim:#1}}
\newcommand{\thmreftwo}[2]      {Theorems \ref{thm:#1} and~\ref{thm:#2}}
\newcommand{\lemlabel}[1]   {\label{lem:#1}}
\newcommand{\lemref}[1]         {Lemma~\ref{lem:#1}}
\newcommand{\algolabel}[1]   {\label{alg:#1}}
\newcommand{\algoref}[1]         {Algorithm~\ref{alg:#1}}
\newcommand{\lemreftwo}[2]      {Lemmas \ref{lem:#1} and~\ref{lem:#2}}
\newcommand{\lemrefthree}[3]    {Lemmas \ref{lem:#1}, \ref{lem:#2}, and~\ref{lem:#3}}
\newcommand{\corlabel}[1]   {\label{cor:#1}}
\newcommand{\corref}[1]         {Corollary~\ref{cor:#1}}
\newcommand{\nonlabel}[1]    {\label{blank:#1}}
\newcommand{\nonref}[1]          {~(\ref{blank:#1})}
\newcommand{\eqlabel}[1]    {\label{eq:#1}}
\newcommand{\eqreff}[1]          {(\ref{eq:#1})}
\renewcommand{\eqref}[1]          {Eq.~\ref{eq:#1}}
\newcommand{\eqreftwo}[2]       {(\ref{eq:#1}) and~(\ref{eq:#2})}
\newcommand{\ineqlabel}[1]    {\label{ineq:#1}}
\newcommand{\ineqref}[1]        {Inequality~(\ref{ineq:#1})}
\newcommand{\ineqreftwo}[2]     {Inequalities (\ref{ineq:#1}) and~(\ref{ineq:#2})}
\newcommand{\invref}[1]         {Invariant~\ref{inv:#1}}
\newcommand{\deflabel}[1]    {\label{def:#1}}
\newcommand{\defref}[1]         {Definition~\ref{def:#1}}
\newcommand{\propref}[1]        {Property~\ref{prop:#1}}
\newcommand{\propreftwo}[2]     {Properties~\ref{prop:#1} and~\ref{prop:#2}}
\newcommand{\proplabel}[1]        {\label{prop:#1}}

\newcommand{\caseref}[1]        {Case~\ref{case:#1}}
\newcommand{\casereftwo}[2]     {Cases \ref{case:#1} and~\ref{case:#2}}
\newcommand{\lilabel}[1]        {\label{li:#1}}
\newcommand{\liref}[1]          {line~\ref{li:#1}}
\newcommand{\Liref}[1]          {Line~\ref{li:#1}}
\newcommand{\lirefs}[2]         {lines \ref{li:#1}--\ref{li:#2}}
\newcommand{\Lirefs}[2]         {Lines \ref{li:#1}--\ref{li:#2}}
\newcommand{\lireftwo}[2]       {lines \ref{li:#1} and~\ref{li:#2}}
\newcommand{\lirefthree}[3]     {lines \ref{li:#1}, \ref{li:#2}, and~\ref{li:#3}}
\newcommand{\exref}[1]          {Exercise~\ref{ex:#1}}
\newcommand{\princref}[1]       {Principle~\ref{prop:#1}}

\newcommand{\obslabel}[1]   {\label{obs:#1}}
\newcommand{\obsref}[1]         {Observation~\ref{obs:#1}}


\newcommand{\resultref}[1]         {Result~\ref{result:#1}}
\newcommand{\resultlabel}[1]   {\label{result:#1}}
\newcommand{\resultreftwo}[2]      {Results~\ref{result:#1} and~\ref{result:#2}}
\newcommand{\resultrefthree}[3]    {Results~\ref{result:#1}, \ref{result:#2}, and~\ref{result:#3}}
\newcommand{\resultrefthrough}[2]      {Results~\ref{result:#1}-\ref{result:#2}}


%%% Local Variables:
%%% mode: latex
%%% TeX-master: "main.tex"
%%% End:


